\chapter{Evaluatie}
Na de ontwerp- en implementatiefase was het tijd om de gecreëerde applicatie te beoordelen en testen.
In het komende hoofdstuk is het dan ook de bedoeling om de applicatie te onderwerpen aan verschillende tests die enkele aspecten in de applicatie gaan testen.
Hierbij is het de bedoeling om te achterhalen wat wel en niet mogelijk is met de huidige versie van de applicatie.
Na het testen van de applicatie, wordt nagedacht voor mogelijke uitbreidingen voor de applicatie.

\section{Testen}
%%% TODO testen
\subsection{Cross-platform host}
\paragraph{Doel}
Het doel van deze test is het uittesten hoeverre de geschreven applicatie uitvoerbaar is op verschillende besturingssystemen.
De applicatie zelf is volledig geschreven in Python met modules die cross-platform zijn.

\paragraph{Scenario}
Om uit te testen of de applicatie cross-platform is, wordt een virtuele machine aangemaakt met de nodige programma's.
Als guest-besturingssysteem voor deze virtuele machine, wordt Linux Mint gekozen. 
Docker, Mysql workbench en wxPython worden geïnstalleerd en met hulp van pip worden de docker-py en dill modules aan python toegevoegd waardoor het systeem klaar is om getest te worden.
De databank werd aangemaakt en de code werd overgebracht naar de virtuele machine.

Tijdens de test is het de bedoeling om alle functionaliteiten van de applicatie uit te testen.
De verschillende docks moeten correct subscriben bij de broker en de nieuwe field dock moet zonder fouten worden toegevoegd aan de databank.
Vervolgens wordt getest of een installer kan aangemaakt worden, verscheept worden naar de field dock en geïnstalleerd kan worden.
Er wordt een installer aangemaakt die bestaat uit 3 verschillende pakketten: één framework pakket, één optioneel pakket en één niet-optioneel pakket.
De twee niet-framework pakketten bevatten een eenvoudig script die output genereert.
Het framework pakket bevat een Grafische User Interface gemaakt met wxPython.

De test is geslaagd als alle verschillende handelingen in de goede volgorde overlopen, ondertussen geen fouten optreden en op het einde het framework opgestart kan worden.

\paragraph{Uitvoering}


\subsection{Cross-platform containers}
\paragraph{Doel}
Naast het controleren of de applicatie zelf cross-platform is, is het belangrijk om te achterhalen of de applicatie om kan met Windows containers.
Het doel van deze test is dan ook om een installer te creëren specifiek bedoelt voor deze Windows containers.

\paragraph{Scenario}
Om gebruik te kunnen maken van Windows containers, moeten enkele vereisten voltooid worden.
Met hulp van \citet{windowsContainers} is het mogelijk om een field dock op te zetten die om kan gaan met de Windows containers.
De testomgeving bestaat, naast de enkele field dock, verder nog uit één release dock.
Deze wordt gebruikt om een installer te creëren die bestaat uit alle nodige pakketten om het Python testraamwerk werkende te krijgen.
Het framework pakket zal dan ook bestaan uit het daadwerkelijke testraamwerk.
Op deze wijze is het mogelijk voor Televic om te achterhalen welke stappen allemaal gezet zijn en welke nog moeten gezet worden om de applicatie zo snel mogelijk in gebruik te nemen.

\paragraph{Uitvoering}

\subsection{Meerdere clients}
\paragraph{Doel}
Voor deze test is het de bedoeling om te achterhalen of de applicatie meerdere clients kan ondersteunen.
Het doel is dan ook een omgeving op te zetten waarin verschillende clients aanwezig zijn.

\paragraph{Scenario}
De omgeving die gebruikt gaat worden bestaat uit één laptop die zowel gebruikt wordt als release dock en als field dock en twee andere computers die gebruikt worden als field docks.
Op de ieder computer is Docker, Python, wxPython, docker-py en dill aanwezig zodat de geschreven code bruikbaar is.

Tijdens de test wordt op één release dock een installer aangemaakt en klaargemaakt voor release.
De installer bestaat uit 2 verschillende pakketten: één framework pakket en één optioneel pakket.
Beide pakketten zullen bestaan uit enkele scripts die output zullen generen wanneer deze worden uitgevoerd.
Naast de release dock zullen 3 field docks toegevoegd worden aan het netwerk.

Om een geslaagde test te verkrijgen, moet de installer bij alle field docks toekomen en correct geïnstalleerd worden.
Dit moet op het einde van het proces zichtbaar zijn in zowel het overzicht op de release dock als in de databank.

\paragraph{Uitvoering}
Na het instellen van de nodige parameters, was het tijd om de nodige systemen op te starten.
Het duurde dan ook niet lang om drie torens te beschrijven die elk bestaan uit één hardware component.
Deze opstellingen werden vervolgens naar het release dock gestuurd en opgeslagen in de databank.

De volgende stap bestaat uit het beschrijven van een simpele installer bestaande uit twee pakketten en deze klaar te maken voor release.


\subsection{Meerdere servers}
\paragraph{Doel}
Een volgende test bestaat uit het testen van de functionaliteiten als meerdere servers aanwezig zijn.
Het doel van de test is te achterhalen wat er gebeurd als meerdere servers aanwezig zijn.

\paragraph{Scenario}
Na het testen of het mogelijk is om meerdere clients te ondersteunen, wordt nu getest of meerdere servers ondersteunt wordt.
Dit gebeurt wederom aan de hand van een virtuele machine met Windows 10.
Iedere virtuele machine bevat de server code samen met een databank om alle gegevens in op te slaan.

Gedurende de test worden 3 release docks aangemaakt en één field dock.
In het eerste deel van de test zal iedere release dock apart een installer creëren en deployen.
De installers zullen bestaan uit één pakket, namelijk een framework pakket.
Het pakket bevat een simpel script dat uitgevoerd wordt.
Voor ieder release dock zal dit script lichtjes verschillen.
Het tweede deel van de test bestaat uit het releasen van 3 verschillende installers op eenzelfde moment.
%%% TODO hier nog iets kleins bijschrijven?

Om de test te doen slagen is het nodig dat: iedere installer moet goed toekomen bij het field dock, de installers moeten in de correcte volgorde geïnstalleerd worden, alle databanken moeten geüpdatet worden en in de verschillende aanpassingen moeten in volgorde toegevoegd worden aan de databank.

\paragraph{Uitvoering}
Met de huidige versie van de applicatie is het mogelijk om drie verschillende servers op te starten en te verbinden met de broker.
Hierbij moet wel rekening gehouden worden met het feit dat het deployment\_server script naast een release dock ook een broker opstart.
Om dit niet te gebruiken moet een boolean op false gezet worden zodanig dat deze code niet wordt uitgevoerd.

Een eerste fase uit de test bestaat uit het correct configureren van een client in het systeem.
Deze wordt correct uitgevoerd en na de beschrijving van de toren worden alle nodige databases aangepast.
Vervolgens wordt op één van de release docks een installer aangemaakt die bestaat uit een simpel script die output genereert.

%%% SHIT probleem -> release wordt niet doorgegeven aan de verschillende release docks -> daardoor niet in databank en niet mogelijk om data van client up te daten


\subsection{Slecht werkend pakket}
\paragraph{Doel}
De geschreven applicatie bevat enkele methodes om te controleren of het installatieproces foutloos is verlopen.
Mocht dit niet het geval zijn, dan wordt de foutieve container in quarantaine geplaatst.
Het doel van deze test is het uittesten of dit wel degelijk gebeurt.

\paragraph{Scenario}
Om deze eigenschap van de applicatie uit te testen wordt een simpele omgeving gerealiseerd waarin getest kan worden.
Eén release dock en field dock worden opgestart in het begin.
Vervolgens wordt een installer gemaakt die bestaat uit twee pakketten.
Het eerste pakket wordt gebruikt om een bestand aan te maken dat nodig is voor het framework pakket.
Het eerstgenoemde pakket zal weliswaar geen bestand aanmaken.

Om de test te laten slagen, moet de test van het niet-framework pakket registeren dat het bestand niet is aangemaakt.
Op het einde van het installatieproces moet de container in quarantaine geplaatst worden.
Zo is het mogelijk om na het installatieproces te controleren wat het probleem is in de container.

\paragraph{Uitvoering}


\subsection{Netwerk monitoring}
\paragraph{Doel}
Als laatste test wordt nagegaan hoe het netwerkverkeer eruit ziet tijdens het releasen van een nieuwe installer.

\paragraph{Scenario}
Deze test wordt gebruikt om te achterhalen hoeveel en wat voor netwerkverkeer gecreëerd wordt door de applicatie.
Het monitoren van het netwerk wordt uitgevoerd tijdens de meerdere client test.
De test produceert voldoende netwerk verkeer om een idee te krijgen van de hoeveelheid verkeer die zal geproduceerd worden.
De volledige communicatie tussen de verschillende docks wordt opgenomen met Wireshark voor later een analyse op uit te voeren.

\paragraph{Uitvoering}


\section{Uitbreidingen}
%%% TODO goede en slechte punten opsommen
%%% SWOT ANALYSE

%geen beveiliging aanwezig

%nog niet volledige communicatie tussen fielddocks aanwezig