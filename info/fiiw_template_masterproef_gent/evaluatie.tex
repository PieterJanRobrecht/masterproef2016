\chapter{Evaluatie}
Na de ontwerp- en implementatiefase was het tijd om de gecreëerde applicatie te beoordelen en testen.
In het komende hoofdstuk is het dan ook de bedoeling om de applicatie te onderwerpen aan verschillende tests die enkele aspecten in de applicatie gaan testen.
Hierbij is het de bedoeling om te achterhalen wat wel en niet mogelijk is met de huidige versie van de applicatie.
Na het testen van de applicatie, wordt nagedacht voor mogelijke uitbreidingen voor de applicatie.

\section{Testen}
%%% TODO testen
\subsection{Cross-platform}
\paragraph{Doel}
Het doel van deze test is het uittesten hoeverre de geschreven applicatie uitvoerbaar is op verschillende besturingssystemen.
De applicatie zelf is volledig geschreven in Python met modules die cross-platform moeten zijn.

\paragraph{Scenario}


\paragraph{Uitvoering}


\subsection{Meerdere clients}
\paragraph{Doel}
Voor deze test is het de bedoeling om te achterhalen of de applicatie meerdere clients kan ondersteunen.
Het doel is dan ook 

\paragraph{Scenario}


\paragraph{Uitvoering}


\subsection{Meerdere servers}
\paragraph{Doel}


\paragraph{Scenario}


\paragraph{Uitvoering}



\subsection{Netwerk monitoring}
\paragraph{Doel}


\paragraph{Scenario}


\paragraph{Uitvoering}


\section{Uitbreidingen}
%%% TODO goede en slechte punten opsommen


%geen beveiliging aanwezig

%nog niet volledige communicatie tussen fielddocks aanwezig