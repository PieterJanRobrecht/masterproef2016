\chapter{Conclusie}
%zeggen of doel bereikt is 
Het doel van deze thesis was het ontwerpen van een oplossing voor het complexe installatieproces en updateproces.
Hiernaast moet rekening gehouden worden met een toenemend aantal gebruikers en software pakketten die geïnstalleerd moeten worden en bij de gebruiker moeten geraken.
Hierbij was het belangrijk om een oplossing te voorzien die zowel schaalbaar als flexibel is.

Om dit probleem op te lossen wordt gebruik gemaakt van de software dock architectuur die de basis vormt voor de applicatie.
Door deze architectuur te combineren met een software packager waarvan het architectuur gebaseerd is op dat van het Qt installer framework en met Docker, was het mogelijk om een flexibele en schaalbare oplossing te voorzien.
De implementatie van het ontwerp vormt een goede basis die Televic kan gebruiken om uit te breiden en aan te passen.

%zeggen wat in de toekomst nog kan komen
Er werd reeds al aangegeven dat het geleverde werk een goede basis vormt voor Televic.
Dit is echter nog geen volwaardig eindproduct aangezien verscheidene functionaliteiten nog niet beschikbaar zijn.
Er moeten nog verscheidene stappen gezet worden om een totaal oplossing af te kunnen leveren maar de grootste stappen zijn reeds al gezet.