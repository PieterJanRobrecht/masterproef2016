\chapter{Analyse en ontwerp}
Uit de probleemstelling werd het snel duidelijk dat het probleem omtrent het Python testraamwerk complexer is dan op het eerste zicht lijkt.
In dit hoofdstuk zal het probleem verder geanalyseerd worden.
Aan de hand van deze bevindingen gaat een architectuur en structuur ontworpen worden.
Deze vormen de basis voor de implementatie van de demo.

\section{Analyse}
Het probleem van Televic was het volgende:
Televic fabriceert producten die moeten voldoen aan strenge veiligheidsnormen.
Om hun producten hierop te kunnen testen heeft Televic een Python testraamwerk ontworpen waarmee het mogelijk wordt om de producten aan verschillende testscenario's te onderwerpen.
Dit testraamwerk maakt gebruik van een grote set aan drivers en bibliotheken om een correcte werking te garanderen.
Een direct gevolg hiervan is dat het installatieproces op een nieuwe testtoren een uitgebreide klus is.
Hiernaast groeit het aantal gebruikers van het testraamwerk continu samen met het aantal drivers en bibliotheken.
Het doel is om een systeem te ontwikkelen dat Televic kan bijstaan bij het installatie en verspreidingsproces.
Een verdere analyse is wel nodig.
Zo is het mogelijk om een applicatie te ontwikkelen die voldoet aan de initiële probleemstelling maar ook uitbreidbaar is naar de toekomst toe.

%Packager
De probleemanalyse onthulde al snel dat dit probleem onder te verdelen is in verschillende deelproblemen.
Het testraamwerk bestaat uit verschillende componenten, hieronder vallen de drivers en bibliotheken.
Elke component heeft een aparte installatiewijze en moeten sommige componenten voor andere geïnstalleerd worden.
Zo zal Python één van de eerste componenten zijn die geïnstalleerd moet worden.
Hiernaast moeten verscheidene componenten geconfigureerd tijdens het installatieproces aan de hand van een configuratiebestand.
Dit configuratiebestand hoort samen met de testtoren waarop het testraamwerk op geïnstalleerd word.
Door gebruik te maken van additionele software worden verschillen in implementaties, door bijvoorbeeld verschillende programmeertalen, opgevangen.
Een stuk van de applicatie zal dus bestaan uit deze additionele software die instaat voor het inpakken van de componenten.
In de rest van de thesis zal naar dit onderdeel de \emph{packager} heten.
Hiervoor kan beroep gedaan worden op verscheidene technologieën, structuren en architecturen die besproken werden in Sectie~\vref{sec:technologieen}.

%Server
De probleemanalyse onthulde ook dat de verschillende executables verspreidt moeten worden.
Door dit proces te automatiseren, is het mogelijk om waardevolle informatie te verzamelen.
Met deze informatie kunnen rapporten gegeneerd worden over het deployment proces.
In Secties~\ref{sec:deployment} - \ref{sec:caseStudies} werden verschillende problemen maar ook oplossingen besproken die aan de basis liggen voor het ontwerp van dit onderdeel van de applicatie. 
In het vervolg van de thesis zal dit onderdeel (dat zal instaan voor het verspreiden van het testraamwerk maar ook voor de communicatie tussen de producten van het testraamwerk en de gebruikers) vermeld worden als de \emph{deployment server}.

%Environment
In Sectie~\vref{sec:softwareLevenscyclus} werd besproken welke problemen kunnen optreden tijdens het installatieproces.
Deze problemen moeten opgevangen worden om een schaalbare oplossing te bedenken voor Televic.
Om dit op te vangen, kan er gebruik gemaakt worden van één (of meerdere) strategieën die besproken werd in Sectie~\vref{sec:rollback}.
Dit onderdeel van de applicatie vooral aanwezig aan de client-side aangezien dat de plaats is waar het testraamwerk aanwezig zal zijn.
In de loop van de thesis zal naar dit onderdeel verwezen worden als de \emph{deployment environment}.

Na de probleemanalyse is het nu duidelijk dat het werk op te delen valt in drie grote componenten.
Deze drie onderdelen zullen de basis vormen voor de architectuur en zullen gebruikt worden als leidraad.
Het eerste onderdeel zal bestaan uit de packager met als doel het inpakken van de nodige drivers, bibliotheken, \ldots .
Naast de packager is er de deployment server instaat voor het verspreiden van de installers die de packager aflevert.
Aan de client-side zal de deployment environment aanwezig zijn waardoor installatie-complicaties vermindert worden door de installatie te isoleren.
Mocht een rollback nodig zijn, dan kan deze op een eenvoudige manier gebeuren.
In Figuur~\vref{fig:overzichtsDiagram} wordt de algemene structuur van de applicatie weergegeven.
Met behulp van deze basis is het mogelijk om een demo te produceren voor de finale verdediging.

\begin{figure}[!hbt]
\centering
  \begin{tikzpicture}[scale=.9, transform shape]
\tikzstyle{every node} = [circle, minimum size = 2cm, fill=gray!30]
\node[align=center] (a) at (0, 0) {Deployment\\ Server};
\node[shape = rectangle,minimum size = 1.5cm] (packager) at +(25: 3) {Packager};
\node[cylinder, shape border rotate=90, draw,minimum height=3cm,minimum width=2cm] (logger) at +(155: 3) {Database};
\node[align=center] (b) at +(225: 5) {Deployment \\ environment 1};
\node[align=center] (c) at +(270: 3.5) {Deployment \\ environment 2};
\node[align=center] (d) at +(315: 5) {Deployment \\ environment 3};
\foreach \from/\to in {a/b, a/c, a/d}
\draw [<->] (\from) -- (\to);
\draw [<->] (a) -- (packager);
\draw [<-] (a) -- (logger);
\end{tikzpicture}
  \caption{Overzichtsdiagram van de algemene structuur}
  \label{fig:overzichtsDiagram}
\end{figure}

\section{Databank ontwerp}\label{sec:databank}
In de voorgaande Sectie werd aangehaald met welke problemen Televic kampt.
Eén van de problemen is de continue groei van pakketten waar het framework gebruikt van maakt en het aantal gebruikers die het framework gebruiken.
Om dit probleem aan te pakken wordt best een databank ontworpen voor het opslaan van alle cruciale data over zowel het installatieproces en alle gebruikers.
In overleg met Televic werd ervoor gekozen om MySQL te gebruiken als managementsysteem.
Het ontwerp van de databank is terug te vinden in Figuur~\vref{fig:databank}.

\begin{figure}[!ht]
\centering
\makebox[0pt]{\includegraphics[scale=0.7]{afbeelding/databankOntwerp.png}}
\caption{Ontwerp van de databank}
\label{fig:databank}
\end{figure}

De tabellen tower en component dienen om iedere gebruiker (zoals een testtoren) te beschrijven.
Iedere toren heeft een ID, naam en serienummer.
De combinatie van deze drie waarden is uniek binnen het bedrijf en de combinatie kan gebruikt worden als identificatie binnen het systeem.
Deze combinatie draagt geen betekenis.
Om dit op te vangen wordt aan iedere toren een alias gekoppeld waardoor de identificatie voor mensen vlotter kan verlopen.
Elke toren bestaat uit verschillende hardware componenten, zoals voedingen of netwerkkaarten, die nodig zijn om testen uit te voeren.
Iedere component is gemaakt door een bepaalde fabrikant en krijgt van de fabrikant een serienummer.
Vanuit het bedrijf wordt losstaand hiervan een nummer toegekend aan iedere component die gebruikt wordt om de calibratie instellingen te achterhalen.
Iedere hardware component zal gebruik maken van firmware om correct te functioneren.
Naast alle bovengenoemde informatie wordt ook de versie van de firmware opgeslagen.
Door het opslaan van al deze informatie wordt het mogelijk om:
\begin{enumerate}
\item torens van elkaar te onderscheiden
\item te achterhalen welke hardware aanwezig is bij welke toren
\item welke firmware versie draait op welke hardware component
\end{enumerate}

Naast informatie over de gebruikers, wordt er ook informatie over de verschillende installers en pakketten bijgehouden.
Iedere installer is een combinatie van verschillende onderdelen, namelijk een versie van het Python testraamwerk, verscheidene pakketten en een configuratie gedeelte (deze structuur is zichtbaar in Figuur~\vref{fig:installerStructuur}).
Het doel is om iedere testtoren te voorzien van één versie van het Python testraamwerk.
Iedere toren zal dus gekoppeld zijn aan één installer.
Hiernaast zullen sommige pakketten gekoppeld zijn aan hardware componenten.
Zo kan een driver voor een voeding gekoppeld worden aan de voeding zelf die aanwezig is in de hardware component tabel.
De hardware-software afhankelijkheden worden op deze manier bijgehouden.
Bij iedere pakket wordt bijgehouden welk type het is (een executabel, zip bestand, \ldots), de prioriteit voor de installatievolgorde, een korte beschrijving en de release datum.
Naast al deze informatie wordt er ook bijgehouden welke pakketten afhankelijk zijn van elkaar.
Een voorbeeld is hiervan is een testraamwerk pakket en een Python installer pakket.
Het testraamwerk is afhankelijk van Python. 
Bij het verschepen van een installer, waar een testraamwerk in aanwezig is, zal dan ook Python aanwezig moeten zijn.
Zo worden de software-software afhankelijkheden bijgehouden.

\begin{figure}[!ht]
\centering
\makebox[0pt]{\includegraphics[scale=0.5]{afbeelding/installerStructuur.png}}
\caption{Structuur van een installer bestaande uit drie pakketten}
\label{fig:installerStructuur}
\end{figure}

Verder zijn er enkele tabellen aanwezig voor het ondersteunen van testen.
Tijdens en na het installatieproces moet het mogelijk zijn om testen uit te voeren.
Dankzij deze testen is het duidelijk of een bepaald pakket correct werkt en op het einde kan gecontroleerd worden of het volledige testraamwerk correct functioneert.
Doordat er een link wordt bijgehouden tussen een hardware component en een pakket, is het mogelijk om hieruit waardevolle informatie uit te halen.
Zo kan bijvoorbeeld een verband gelegd worden tussen een bepaalde versie van een driver en de firmware die aanwezig is in een hardware component.
Deze informatie kan gebruikt worden om problemen in testtorens te vermijden.

\section{Architectuur}

\subsection{Packager}
De architectuur van de packager wordt gebaseerd op de architectuur en structuur van het Qt installer framework.
Hierbij wordt bedoelt dat één overkoepelende installer wordt geproduceerd die bestaat uit verschillende kleine componenten.
Het Qt installer framework wordt hiervoor niet zelf gebruikt.
Deze keuze werd gemaakt op basis van verschillende argumenten.
%het zorgt voor een beter bereik van de verschillende packages -> kunnen test tussendoor uitvoeren
Door het personaliseren van de packager, is het mogelijk om iedere stap in het deployment proces te personaliseren.
Op deze manier kan na het installeren van een pakket een verbeterde afhandeling plaats vinden.
Een test kan uitgevoerd worden op het einde van de installatie, een handeling die met het Qt installer framework mogelijk is maar moeilijk te realiseren is.
%geen gedoe met docker
Doordat een gepersonaliseerde packager wordt ontworpen, worden problemen met Docker vermeden.
In de deployment environment wordt Docker gebruikt om verscheidene problemen met het deployment proces op te vangen (dit wordt verder besproken).
De Docker omgeving, zoals reeds vermeld in Sectie~\vref{sec:virtualisatie}, gebruikt van LXC.
Het besturingssysteem van de containers aan de client side is hierdoor Linux.
Windows gebruiken als besturingssysteem is mogelijk maar deze optie staat nog altijd in beta schoenen.
De installer geproduceerd door het Qt installer framework moet dus compatibel zijn met Linux.
Dit is mogelijk met het framework maar de productie van de installer moet ook plaatsvinden in een Linux besturingssysteem.
Om dit te realiseren zou Docker gebruikt kunnen worden zodanig dat er een abstractie gedaan wordt van het host besturingssysteem.
Zo'n opstelling creëren, waarbij een container gebruikt wordt met het Qt installer framework in verwerkt, brengt meer werkt en is omslachtiger ten opzichte van het zelf fabriceren van een packager met een gelijkaardige structuur en gelijkaardige functionaliteiten.

%%% TODO schrijven hoe het er dan wel uit gaat zien
De packager gaat een gelijkaardige structuur hebben als het Qt installer framework.
Om de packager te maken wordt Python gebruikt als programmeertaal.
Een installer bestaat uit verschillende pakketten en functioneert als overkoepelend geheel.
Per test framework wordt één installer geassocieerd.
Voor alle drivers/bibliotheken die nodig zijn, worden verschillende pakketten voorzien waarbij één driver hoort bij één pakket.
Ieder pakket bestaat uit twee delen: een data en metadata gedeelte.
Het data gedeelte bevat de effectieve driver/bibliotheek en het metadata gedeelte bevat de nodige beschrijving van het pakket in de vorm van een ``package.json''.
Hiernaast zijn verschillende scripts aanwezig die gebruikt worden voor een gepersonaliseerde installatie.
De structuur van een mogelijke installer is terug te vinden in Figuur~\vref{fig:installerStructuur}.
In het voorbeeld wordt een installer opgebouwd bestaande uit twee delen.
Het eerste deel bestaat uitsluitend uit configuratie bestanden voor de volledige installer (deze bevat bijvoorbeeld de installatie folder) en het tweede deel bestaat uit de pakketten.
De twee onderdelen zijn terug te vinden in de rode rechthoek.
Eén van de packages is het testraamwerk zelf.
Hiernaast zijn er ook twee cruciale pakketten aanwezig die de correcte werking voorzien.
Zo kan Python dienen als een voorbeeld voor deze pakketten.
Voordat het raamwerk geïnstalleerd kan worden, moet eerst Python aanwezig zijn.

Om een dergelijke structuur op te bouwen, werkt de packager nauw samen met de databank waarin alle informatie vervat zit.
Het ontwerp van de databank werd al uitvoerig besproken in Sectie~\ref{sec:databank} en is terug te vinden in Figuur~\ref{fig:databank}.

\subsection{Deployment server}
Het centrale systeem in de architectuur is de deployment server.
Zoals reeds uitgelegd zal dit onderdeel instaan voor het verspreiden van de verschillende installers en functioneren als een verzamelcenter voor alle informatie.
De architectuur van de deployment server wordt gebaseerd op de software dock architectuur die besproken werd in Sectie~\vref{sec:softwareDock} en is terug te vinden in Figuur~\vref{fig:softwareDockAangepast}.

\begin{figure}[!ht]
\centering
\makebox[0pt]{\includegraphics[scale=0.5]{afbeelding/softwareDockAangepast.png}}
\caption{Software Dock Architectuur \citep{hall1999cooperative}}
\label{fig:softwareDockAangepast}
\end{figure}

De software dock architectuur bestaat uit 4 grote componenten, namelijk het release dock, field dock, event service en de agenten.
Het release dock bevindt zich aan de serverzijde en bevat alle software van de packager.
Met hulp van de packager worden verschillende releases geproduceerd.
Als een release klaar is voor deployement wordt een event afgevuurd naar de event service.
Alle agenten die geabonneerd zijn op het gepaste event worden vervolgens op de hoogte gebracht.
Deze gedistribueerde architectuur laat een eenvoudige uitbreiding van het aantal docks toe.
Het tweede type dock dat aanwezig is in de architectuur is de field dock.
De verschillende clients functioneren als een field dock en zullen communiceren met het release dock aan de hand van de event-service.

%%schrijven over hoe de agenten werken en hoe ze gaan werken volgens de handelingen van ORYA
Naast de docks bevat de architectuur agenten.
Deze staan in voor het uitvoeren van allerlei deployment gerelateerde handelingen.
Iedere agent is gekoppeld aan één stap uit de software levenscyclus die besproken werd in Sectie~\vref{sec:softwareLevenscyclus}.
Hiernaast zal aan iedere release van het release dock een subset van alle agenten toegevoegd en verscheept worden naar het field dock.
Zo wordt bijvoorbeeld een agent voorzien die instaat voor het installatieproces.
De agent wordt samen met de release verscheept naar het field dock waarna de agent in actie schiet.
De agent begint met het creëren van een nieuwe Docker container waarin de installer losgelaten kan worden.
Vervolgens zal de agent de installatie aanvangen en zullen de scripts horende bij de pakketten uitgevoerd worden in de container.
Ieder agent zal een bepaalde set van handelingen uitvoeren die overeenkomt met een deployment proces die besproken werd in de ORYA case studie in Sectie~\vref{sec:ORYA}.
Net zoals bij ORYA wordt ieder deployment proces beschreven aan de hand van andere deployment processen en basis activiteiten.
Het creëren van een nieuwe container in de installatie agent wordt gezien als zo'n basis activiteit.
In Figuur~\vref{fig:fieldDock} wordt de algemene structuur van een field dock weergegeven.

\begin{figure}[!ht]
\centering
\makebox[0pt]{\includegraphics[scale=0.5]{afbeelding/fieldDock.png}}
\caption{Structuur van een field dock}
\label{fig:fieldDock}
\end{figure}

Door agenten te gebruiken, een strategie die ook gezien werd in de Atlas case studie in Sectie~\vref{sec:ATLAS}, wordt het mogelijk om alle stappen in de software levenscyclus uniek te behandelen.
Hiernaast kan bij iedere release een andere set van agenten geassocieerd worden waardoor iedere release verder kan gepersonaliseerd worden.
De volledige architectuur is terug te vinden in Figuur~\vref{fig:architectuur}.
In de figuur zijn zowel de packager als de deployment environment toegevoegd.

\begin{figure}[!ht]
\centering
\makebox[0pt]{\includegraphics[scale=0.5]{afbeelding/architectuur.png}}
\caption{Architectuur van het prototype}
\label{fig:architectuur}
\end{figure}

\subsection{Deployment environment}
De deployment environment komt overeen met de field dock in de software dock architectuur.
In de omgeving gaat de installer, afkomstig van de packager, uitgevoerd worden zodanig dat het test framework geïnstalleerd wordt.
Aan dit proces zijn de verschillende problemen verbonden die besproken zijn in Sectie~\vref{sec:softwareLevenscyclus}.
Om de verschillende deployment problemen te vermijden en om ervoor te zorgen dat geen uitgebreide rollback strategieën nodig zijn, wordt een geïsoleerde omgeving voorzien waarin de software gedeployed kan worden. 
Dit wordt gerealiseerd aan de hand van virtualisatie technieken, meer bepaald aan de hand van Docker.
Docker wordt verkozen boven een gewone virtuele machine omdat het uitvoeren van handelingen (zoals opstarten, stoppen, \ldots) op een container minder resources en tijd vraagt in vergelijking met een virtuele machine.
De container wordt vervolgens gebruikt om het testraamwerk in te installeren.
Doordat een virtualisatie techniek wordt gebruikt, wordt het zeer eenvoudig om problemen tijdens het deployment en installatieproces op te vangen.
In Figuur~\vref{fig:flow:install} en Figuur~\vref{fig:flow:rollback} is het duidelijk dat, door het gebruik van Docker, het rollback proces zeer eenvoudig is.
