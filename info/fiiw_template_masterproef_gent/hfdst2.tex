%In de besprekeking ga ik het volgende aanhalen:
%-Aanhalen van de verschillende technologieen
%-Het onderbouwen waarom ik zo te werk ben gegaan -> hier worden de bronnen gebruikt
\chapter{Bespreking}
De structuur die in Hoofdstuk~\ref{hfdst:situering} werd opgebouwd
In de loop van dit hoofdstuk zullen de verschillende technologieën besproken worden die in aanmerking komen.
Naar het einde van het hoofdstuk toe, zal een keuze gemaakt worden en zal voor iedere component de meest geschikte technologie gekozen worden.

\section{Packager}
De eerste component die besproken wordt is de packager.
Zoals reeds besproken, moet tijdens het ontwerpen van deze component rekening gehouden worden met enkele eigenschappen.
%Bron zoeken die aanhaalt waarom ik op een bepaalde manier alles verdeel
Alvorens bepaald wordt welke technologie geschikt is, moet onderzocht worden wat ingepakt moet worden en op welke manier.
Om een correct werkende applicatie te hebben, heeft het Python raamwerk verschillende drivers en bibliotheken nodig.
%bespreken van technologieen
%manier van opdeling bespreken
\citep{packAtlas} gaat op een gelijkaardig manier te werk. 
Met behulp van een "Pacman file" is geweten hoe de ingepakte software moet worden behandelt.


%\begin{figure}[!h]
%\centering
%  % We need layers to draw the block diagram
\pgfdeclarelayer{background}
\pgfdeclarelayer{foreground}
\pgfsetlayers{background,main,foreground}

% Define a few styles and constants
\tikzstyle{sensor}=[draw, fill=blue!20, text width=5em, 
    text centered, minimum height=2.5em]
\tikzstyle{ann} = [above, text width=5em]
\tikzstyle{naveqs} = [sensor, text width=6em, fill=red!20, 
    minimum height=12em, rounded corners]
\tikzstyle{naveqq} = [sensor, text width=6em, fill=red!20, 
    minimum height=2.5em, rounded corners]
\def\blockdist{2.3}
\def\edgedist{2.5}

\begin{tikzpicture}
    \node (packagerServer) [naveqs] {Packager Interface};
    % Note the use of \path instead of \node at ... below. 
    \path (packagerServer.140)+(-\blockdist,0) node (gyros) [sensor] {Rapport};
    \path (packagerServer.-150)+(-\blockdist,0) node (collect) [sensor] {Data Collector};
    
    % Unfortunately we cant use the convenient \path (fromnode) -- (tonode) 
    % syntax here. This is because TikZ draws the path from the node centers
    % and clip the path at the node boundaries. We want horizontal lines, but
    % the sensor and naveq blocks aren't aligned horizontally. Instead we use
    % the line intersection syntax |- to calculate the correct coordinate
    % We could simply have written (gyros) .. (naveq.140). However, it's
    % best to avoid hard coding coordinates
    \path [draw, ->] (collect) -- node [left] {info} 
        (collect.north |- gyros.south);
    \path [draw, <-] (collect) -- node [below] {info} 
        (packagerServer.west |- collect);    
    \node (IMU) [above of=gyros] {Status};
    \path (packagerServer.south west)+(-0.6,-0.4) node (INS) {Deployment Server};
    
    % Now it's time to draw the colored IMU and INS rectangles.
    % To draw them behind the blocks we use pgf layers. This way we  
    % can use the above block coordinates to place the backgrounds   
    \begin{pgfonlayer}{background}
        % Compute a few helper coordinates
        \path (gyros.west |- packagerServer.north)+(-0.5,0.3) node (a) {};
        \path (INS.south -| packagerServer.east)+(+0.3,-0.2) node (b) {};
        \path[fill=yellow!20,rounded corners, draw=black!50, dashed]
            (a) rectangle (b);
        \path (collect.south east)+(0.3,-0.3) node (a) {};
        \path (IMU.north -| collect.west)+(-0.3,0) node (b) {};
        \path[fill=blue!10,rounded corners, draw=black!50, dashed]
            (a) rectangle (b);
    \end{pgfonlayer}
    
    
    
	\node[below = 2cm of packagerServer] (receiver) [naveqq] {Package Receiver};
    % Note the use of \path instead of \node at ... below. 
    \path (receiver.180)+(-\blockdist,0) node (gyros) [naveqq] {Status};
    \path (receiver.270)+(0,-1.5) node (accel) [naveqq] {Unpacker};
    \path (accel.180)+(-\blockdist,0) node (installer) [naveqq] {Installer};
    % Unfortunately we cant use the convenient \path (fromnode) -- (tonode) 
    % syntax here. This is because TikZ draws the path from the node centers
    % and clip the path at the node boundaries. We want horizontal lines, but
    % the sensor and naveq blocks aren't aligned horizontally. Instead we use
    % the line intersection syntax |- to calculate the correct coordinate
    \path [draw, ->] (gyros) -- node [left] {data} 
        (gyros |- collect.south) ;
    \path [draw, ->] (receiver) -- node [right] {New package} 
        (receiver |- accel.north) ;
	\path [draw, ->] (accel) -- node [above] {} 
        (installer.east |- accel) ;
	\path [draw, ->] (installer) -- node [right] {OK} 
        (installer |- gyros.south) ;
    \path [draw, ->] (packagerServer) -- node [above] {} 
        (packagerServer.south |- receiver.north);        
    % We could simply have written (gyros) .. (naveq.140). However, it's
    % best to avoid hard coding coordinates
    \path (accel.south west)+(-0.6,-0.4) node (INS) {Deployment Environment};


    % Now it's time to draw the colored IMU and INS rectangles.
    % To draw them behind the blocks we use pgf layers. This way we  
    % can use the above block coordinates to place the backgrounds   
    \begin{pgfonlayer}{background}
        % Compute a few helper coordinates
        \path (gyros.west |- receiver.north)+(-0.5,0.3) node (a) {};
        \path (INS.south -| receiver.east)+(+0.3,-0.2) node (b) {};
        \path[fill=yellow!20,rounded corners, draw=black!50, dashed]
            (a) rectangle (b);
    \end{pgfonlayer}
    
    
	\node[right= 4cm of packagerServer.north, anchor=north](mapper) [naveqq] {Mapper};
    % Note the use of \path instead of \node at ... below. 
    \path (mapper.-90)+(0,-1.5) node (interface) [naveqq] {Packager Interface};
    \path (mapper)+(3,0) node (accel) [naveqq] {Package Producer};
    
    % Unfortunately we cant use the convenient \path (fromnode) -- (tonode) 
    % syntax here. This is because TikZ draws the path from the node centers
    % and clip the path at the node boundaries. We want horizontal lines, but
    % the sensor and naveq blocks aren't aligned horizontally. Instead we use
    % the line intersection syntax |- to calculate the correct coordinate
%    \path [draw, ->] (gyros) -- node [above] {$\vc{\omega}_{ib}^b$} 
%        (test.west |- gyros) ;
%    % We could simply have written (gyros) .. (naveq.140). However, it's
%    % best to avoid hard coding coordinates
    \path [draw, ->] (mapper) -- node [above] {} 
        (accel.west |- mapper);
    \path [draw, ->] (accel.south) -- (interface.east);
    \path [draw, ->] (interface) -- node [above] {} 
        (packagerServer.east |- interface);
	\path (mapper.90)+(0,1.5) node (db) {Database};
    \path [draw, -] (db) -- node [above] {} 
        (db.south |- mapper.north);
    \path (interface.south east)+(0.3,-0.4) node (INS) {Packager};


    % Now it's time to draw the colored IMU and INS rectangles.
    % To draw them behind the blocks we use pgf layers. This way we  
    % can use the above block coordinates to place the backgrounds   
    \begin{pgfonlayer}{background}
        % Compute a few helper coordinates
        \path (interface.west |- mapper.north)+(-0.5,0.3) node (a) {};
        \path (INS.south -| accel.east)+(+0.3,-0.2) node (b) {};
        \path[fill=yellow!20,rounded corners, draw=black!50, dashed]
            (a) rectangle (b);
    \end{pgfonlayer}
\end{tikzpicture}
%  \caption{Blok diagram met alle componenten}
%  \label{fig:overzichtBlok}
%\end{figure}

\section{Packager}
Het eerste onderdeel van de applicatie is de packager.
Zoals in Sectie~\ref{sec:probleem} is uitgelegd, moet de packager instaan voor het samenvoegen van alle drivers en bibliotheken.
%
%BlokSchema met de algemene structuur van de packager
%

%In het vorig hoofdstuk hebben we naar deze tekst verwezen\label{verwijzing}.